\documentclass[12pt,a4paper]{article}

\usepackage[utf8]{inputenc}
\usepackage[T1]{fontenc}
\usepackage{lmodern}
\usepackage[margin=2.5cm]{geometry}

\begin{document}

\begin{abstract}
The project \textbf{GROCOMO-CFT} (\emph{GROupoid COvariance and Multilocal Observables in Classical Field Theories on curved backgrounds}) aims to construct a geometric framework for classical field theories on curved spacetimes by introducing \emph{Lie groupoids} to model local, observer-dependent symmetries and by extending multisymplectic geometry to \emph{multilocal observables}. 

First, we will analyze the \emph{Poincaré groupoid} $O(TM,g)\rightrightarrows M$ and its Spin extension, clarifying its relation to the frame groupoid $GL(TM)$ and to jet prolongations $J^1(\mathrm{Pair}(M))$. This provides the appropriate replacement of global Poincaré symmetry on curved Lorentzian manifolds. 

Second, we will formulate a \emph{groupoid gauge theory of gravitation} by defining connections on the Lie algebroid of the Poincaré groupoid. The associated curvature and torsion will be related to tetrads and spin connections, and the formalism will be compared with Einstein–Cartan and teleparallel gravity. 

Third, we will extend the algebraic structures of multisymplectic geometry to include \emph{multilocal observables}. This involves defining $L_\infty$-algebras associated with groupoid symmetries and constructing \emph{homotopy comomentum maps}. Explicit models, including scalar fields and selected sectors of the Standard Model on curved backgrounds, will serve as test cases.

The technical goal is to provide a covariant formalism that generalizes Noether’s theorem, incorporates fermionic fields through spin extensions, and enables a coherent treatment of observables beyond the local setting. The framework combines tools from differential geometry, higher algebra, and field theory, and builds directly on recent advances such as groupoid-based formulations of conservation laws and higher-gauge approaches to gravity.
\end{abstract}

\end{document}


The project GROCOMO-CFT (GROupoid COvariance and Multilocal Observables in Classical Field Theories on curved backgrounds) develops a new geometric framework for classical field theories on curved spacetimes by introducing Lie groupoids to model local, observer-dependent symmetries and by extending multisymplectic geometry to multilocal observables. The aim is to overcome the limitations of current formulations of the Standard Model, which rely on global Poincaré symmetry valid only on flat spacetime.

The first objective is the construction and analysis of the Poincaré groupoid, together with its Spin extension required to accommodate fermionic matter. The relation between the frame groupoid GL(TM), jet prolongations of the pair groupoid J¹(Pair(M)), and the covariance properties of observer-dependent charts will be clarified, providing the appropriate generalization of global Poincaré invariance to curved Lorentzian manifolds.

The second objective is the development of a groupoid gauge theory of gravitation. Connections on the Lie algebroid of the Poincaré groupoid will be defined, and the resulting notions of curvature and torsion will be compared with the tetrad and spin-connection formalism. This approach allows new tools to deal with Einstein–Cartan and teleparallel regimes, with translations kept as external symmetries rather than promoted to internal ones.

The third objective is to extend the algebraic structures of multisymplectic geometry to multilocal observables. This involves defining L-infinity algebras of observables equivariant under groupoid symmetries and constructing homotopy comomentum maps. Explicit case studies, including scalar fields and selected Standard Model sectors on fixed curved spacetimes, will test the consistency of the framework.

By combining differential geometry, higher algebra, and field theory, the project aims to establish a rigorous, covariant formalism that generalizes Noether’s theorem, provides a coherent treatment of multilocal observables, and integrates gravity and matter fields into a groupoid-based setting.